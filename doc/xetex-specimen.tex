\documentclass[letterpaper]{article}

\input notesdef
\usepackage[unicode]{hyperref}
\hypersetup{
  pdfborder=0 0 0,
  pdftitle={XeTeX-specimen},
  pdfauthor={Clerk Ma, AlphaZTX},
  pdffitwindow,
  pdfkeywords={XeTeX, font, OpenType},
  pdfcreator=xdvipdfmx,
  pdfproducer=XeTeX with specimen library,
  colorlinks,
  linkcolor=black,
  urlcolor=titlecolor!75!black,
}
\input hyperdef

\title{\color{titlecolor}\xetexspecimen}
\author{Clerk Ma \\ 文档作者:AlphaZTX}
\date{2022-10-15}



\begin{document}

\maketitle
\thispagestyle{empty}

\section*{简介}
\xetex{} 在 Windows 系统下默认使用 \lib{fontconfig} 库来获取可用的 OpenType 字体。
\lib{fontconfig} 是 Linux 系统下的字体管理库,移植到 Windows 中会出现速度变慢的现象。
\xetexspecimen{} 专为 Windows 下的字体获取做出了优化,在字体获取速度方面较原始的 
\xetex{} 有较大的提升。

\xetexspecimen{} 不依赖于 \lib{fontconfig},但需要 Python 3。

%\clearpage
\begingroup\parskip0pt\relax
\tableofcontents\endgroup
%\clearpage

\section{安装 \xetexspecimen}
首先,从 \linktext{https://github.com/clerkma/xetex-specimen/releases}%
{GitHub Releases} 中下载编译好的压缩包,例如 \file{build-1015.zip}。
在下载的压缩包中,你会看到类似于下面的目录结构:
\begin{flushleft}
\leftskip2em\lineskip0pt\lineskiplimit0pt
\linespread{1}\selectfont\ttfamily
build-DATE\\[2pt]
\hskip1em├ dll-x86\\[-.4pt]
\hskip1em├ exe-a64\\[-.4pt]
\hskip1em├ exe-x64\\[-.4pt]
\hskip1em└ exe-x86\\[-.4pt]
\end{flushleft}
其中,\file{dll-x86} 路径下的文件是动态链接库 \file{xetex.dll},这里的 
\file{xetex.dll} 不依赖于 \lib{fontconfig} 库。

\file{exe-a64} 路径中包含了 ARM 64 架构下的 \xetex{} 程序本体;
\file{exe-x64} 路径中包含了 x64(AMD 64)架构下的 \xetex{} 程序本体;
\file{exe-x86} 路径中包含了 x86 架构下的 \xetex{} 程序本体。

\paragraph{替换 \texlive{} 中的动态库}
将 \file{dll-x86} 下的 \file{xetex.dll} 复制到 \texlive{} 中的 \file{bin} 目录下
(例如 \file{C:\textbackslash texlive\textbackslash 2022\textbackslash bin},
具体的路径取决于安装位置),即完成动态库的替换。

如果你不准备长期使用 \xetexspecimen,也可以保留 \file{bin} 路径下原有的 
\file{xetex.dll} 以便随时替换回来。

\paragraph{替换 \xetex{} 本体}
对于不同架构的 Windows 系统,也可以选择不同系统架构下专用的 \xetex{} 程序本体。
将与当前系统架构对应的 \xetex{} 本体(\file{xetex.exe})从下载的压缩包复制到 
\file{bin} 路径下,即完成 \xetex{} 本体的替换。注意,这一步是可选的。你也可以
选择使用 \texlive{} 中原有的 \file{xetex.exe}。

\paragraph{刷新字体数据库}
在 \linktext{https://github.com/clerkma/xetex-specimen/libspecimen}%
{GitHub: lib\lib{specimen}} 中,你可以找到一份 Python 脚本 \file{gen-fontdb.py}。
将这段 Python 脚本下载或保存到任意路径下,首先检查 \file{gen-fontdb.py} 
最后几行中 \texlive{} 的路径是否与你选择的 \texlive{} 安装路径是否一致,
若不一致,则需修改。例如,假定你的 \texlive{} 安装在 D 盘中,则需将 
\file{gen-fontdb.py} 的最后几行由以下代码:
\begin{Verbatim}[numbers=left,firstnumber=229]
if __name__ == "__main__":
    user_path_list = [
        r"C:\texlive\2022\texmf-dist\fonts\opentype",
        r"C:\texlive\2022\texmf-dist\fonts\truetype",
        r"C:\windows\fonts"
    ]
    parse(user_path_list)
\end{Verbatim}
改为:
\begin{Verbatim}[numbers=left,firstnumber=229]
if __name__ == "__main__":
    user_path_list = [
        r"D:\texlive\2022\texmf-dist\fonts\opentype",
        r"D:\texlive\2022\texmf-dist\fonts\truetype",
        r"C:\windows\fonts"
    ]
    parse(user_path_list)
\end{Verbatim}
如果你还想使用更多路径,可以在上面的 \verb|user_path_list| 中添加需要的路径。

最后,使用 Python 执行 \file{gen-fontdb.py}。例如:
\begin{Verbatim}
python gen-fontdb.py
\end{Verbatim}
等候一段时间。生成 JSON 文件后,即完成全部安装过程。

\section{使用}
\xetexspecimen{} 和 \texlive{} 中原有的 \xetex{} 的使用方法完全相同。
编译命令仍然是 \verb|xetex| 和 \verb|xelatex|。\xetex{} 下的 \cs{font} 
原语可正常使用,\xelatex{} 下的 \pkg{fontspec}、\pkg{unicode-math}、
\pkg{xeCJK} 等宏包也可正常使用。举一组例子:
\begin{example}
\font\0="Inter"
\0 The quick brown fox jumps over the lazy dog.            \par
\font\1="Libertinus Serif/I" at 12pt
\1 The quick brown fox jumps over the lazy dog.            \par
\font\2="[TeXGyrePagella-Regular.otf]:+smcp" scaled 1600
\2 The quick brown fox jumps over the lazy dog.            \par
\font\3="Latin Modern Roman/S=22:color=0000FF,mapping=tex-text"
\3 A blue quoted em-dash: ``---''.
\end{example}
\begin{example}
\font\4="[STIXTwoMath-Regular.otf]"\relax \4
\the\XeTeXcountglyphs\4            \qquad
\XeTeXglyph 1699                   \qquad
\XeTeXglyphname\4 1699             \qquad
\the\XeTeXglyphindex"uni222B.dsp"  \relax
\end{example}
以上代码中的部分命令为 \xetex{} 的原语,详见 
\linktext{https://ctan.org/pkg/xetexref}{\xetex{} 的文档}。

对于 \xelatex,我们也举一组例子:
\begin{example}
% \usepackage{fontspec}
\fontspec{Latin Modern Sans}[Color=008800]
The quick brown fox jumps over the lazy dog.
\end{example}
\begin{example}
% \usepackage{unicode-math}
\[ ∫_{-∞}^∞ δ(x) \, dx = 1. \]
\end{example}
\begin{example}
% \usepackage{xeCJK}
他说:“我能吞下玻璃而不伤身体。”
\end{example}





\end{document}